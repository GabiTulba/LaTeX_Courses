\documentclass{article}
\usepackage[utf8]{inputenc}
\usepackage{mathtools, nccmath}
\usepackage{mathabx}
\usepackage{mathrsfs}
\usepackage{natbib}
\usepackage{graphicx}
\usepackage{cancel}
\usepackage{upquote}
\usepackage{pgfplots}
\usepackage{amssymb}
\pgfplotsset{width=10cm,compat=1.9}
\usepgfplotslibrary{external}
\tikzexternalize
\usepackage{hyperref}
\usepackage{xparse}
\usetikzlibrary{through}

\usepackage{tikz}
\usetikzlibrary{angles,quotes}

\newcommand{\handC}{\mathscr{C}}
\newcommand*{\QEDA}{\hfill\ensuremath{\blacksquare}}%
\title{Mate 1: Curs \#2}
\author{Profesor: Radu Gologan}
\date{3 Octombrie 2019}

\DeclarePairedDelimiterX{\set}[1]{\{}{\}}{\setargs{#1}}
\NewDocumentCommand{\setargs}{>{\SplitArgument{1}{;}}m}
{\setargsaux#1}
\NewDocumentCommand{\setargsaux}{mm}
{\IfNoValueTF{#2}{#1} {#1\,\delimsize|\,\mathopen{}#2}}%{#1\:;\:#2}

\begin{document}
    
    \maketitle
    \section{Latice}
        Definitie: $(X,\leq)$ se numeste \textbf{latice completa} daca $\forall A \subseteq X\implies \exists sub_A,inf_A\in X$
        $f:X\rightarrow X$ este o functie monotona crescatoare(descrescatoare) daca $x\leq y \implies f(x)\leq(\geq)f(y)$\\
        Exemplu:\\
        Fie $M$ o multime, atunci $\mathscr{P}(M)$ este o latice completa relativ la relatia de incluziune $\subseteq$, cu $sup_{A_i} =
        \displaystyle{\bigcup_{i}} A_i$ si $inf_{A_i}=\displaystyle{\bigcap_{i}}A_i,\ A_i\in \mathscr{P}(M)$\\
        Teorema (Tarski):\\
        Fie $(X,\leq)$ o latice completa si $f:X\rightarrow X$ o functie monotona. Atunci $f$ admite un punct fix: $\exists u \in X\ a.i.\ f(u)=u$\\
        Demonstratie:\\
        Fie $A\subseteq X\ a.i\ A=\set{x\in X;x\leq f(x)}\ si\ u = sup_A$.\\ 
        Din faptul ca $x\in A \implies x\leq f(x)$, dar $f$ este monotona $\implies f(x)\leq f(f(x)) \implies f(x) \in A\ \forall x\in
        A$ \hfill (1)\\
        $\forall x\in A,\ x\leq u$ si f monotona $\implies f(x) \leq f(u)\implies x\leq f(x)\leq f(u)\implies\ f(u)$ majorant pentru $A$, dar $u=sup_A\implies u
        \leq f(u)$ \hfill (2)\\
        Din faptul ca $u\leq f(u) \implies u\in A$, dar din (1) rezulta ca $f(u)\in A\ si\ u=sup_A \implies f(u)\leq u$ \hfill (3)\\
        Din (2) si (3) $\implies f(u)=u$ \hfill \QEDA\\ \\
        Consecinta a teoremei Tarski - Teorema lui Bernstein:\\
        Fie $A,B$ doua multimi, astfel incat $\exists f:A\rightarrow B,\ g:B\rightarrow A$ doua functii injective. Atunci exista o functie $h:A\rightarrow B$
        bijectiva.\\
        Demonstratie:\\
        Fie $\phi:\mathscr{P}(A)\rightarrow \mathscr{P}(A)$ o functie. $\phi$ este o functie crescatoare(descrescatoare) daca $\phi$ are urmatoarea proprietate
        $\forall U,V\in \mathscr{P}(A)\ a.i.\ U\subseteq V \implies \\  \phi(U) \subseteq (\supseteq) \phi(V)$ \hfill (4)\\
        Fie $A$ o multime si $\phi:\mathscr{P}(A)\rightarrow \mathscr{P}(A),\ \phi(X) = A\setminus X$ o functie. Pentru $U,V \in \mathscr{P}(A), U \subseteq V
        \implies \phi(U) \supseteq \phi(V),\ din\ (1) \implies \phi$ este
        descrescatoare.(5)\\ 
        Pentru $f:A\rightarrow B$, vom nota pentru $X\in \mathscr{P}(A), f(X) = \set{f(x);x\in X}$.\\ \\
        Fie $f:A\rightarrow B$ si $g:B\rightarrow A$ doua funictii injective.\\ Fie $S:\mathscr{P}(A) \rightarrow \mathscr{P}(A),\ S(X) = g(B\setminus f(A\setminus
        X))$. Din (4) si (5) rezulta ca $S$ este crescatoare. Deoarece $S$ este monotona, din teorema lui Tarski rezulta ca $\exists C\in \mathscr{P}(A)\ a.i.\
        S(C)=C \implies \forall x,\ x \in C \iff x \in g(B\setminus f(A\setminus C)$ \hfill (6).\\ \\
        Fie $R = \set{(x,y)\in A\times B; x\ \cancel{\in}\ C\ si\ y=f(x),\ sau\ x\in C\ si\ x=g(y)}$ o functie definita pe A, cu valori in B, vom arata ca R este o functie bijectiva.\\
        Fie $y \in B$. Daca $y \in f(A\setminus C) \implies \exists x \in A\setminus C\ a.i.\ f(x) = y$. Daca $y\ \cancel{\in}\ f(A\setminus C) \implies y\in
        B\setminus f(A\setminus C) \implies \exists x =g(y) \in C$. Rezulta ca $R$ este surjectiva \hfill (7)\\
    \section{Numarabilitate}
        Definitie:\\
        Fie $A,B$ doua multimi, supunem ca $A$ si $B$ sunt \textbf{echivalente} $\iff \exists f:A\rightarrow B$ o functie bijectiva.\\
        Definitie:\\
        O multime $A$ echivalenta cu $\mathbb{N}$ se numeste \textbf{numarabila}.\\
        Notam $card(A)$ \textbf{cardinalul} lui $A$ = clasa de echivalenta a lui A fata de relatia de echivalenta cu \mathbb{N}.\\
        Notam $card(\mathbb{N})=\aleph_0$\\
        Altfel spus, $A$ este numarabila $\iff \exists (a_n)_{n\geq0}\ a.i.\ \forall x\in A\ \exists i\in \mathbb{N}\ a.i.\ a_i = x$\\ \\
        Propozitie: daca $A$ si $B$ sunt numarabile $\implies A\times B$ este numarabila.\\
        Demonstratie:\\
        Este suficient sa demonstram ca $\mathbb{N} \times \mathbb{N}$ este numarabila, intrucat daca $A$ si $B$ sunt echivalente cu $\mathbb{N} \implies
        A\times B$ este echivalenta cu $\mathbb{N} \times \mathbb{N}$.\\
        $\mathbb{N} \times \mathbb{N} = \set{(n,m);n,m\in\mathbb{N}}$\\
        Fie $f:\mathbb{N} \times \mathbb{N} \rightarrow \mathbb{N},\ f((i,j)) = \frac{(i+j)\cdot(i+j-1)}{2}+i$.\\ Vom demonstra ca $f$ este bijectiva.\\
        Se observa ca daca $a+b=k\implies f((a,b))\in [\frac{k(k-1)}{2},\frac{k(k+1)}{2})$ si oricare doua astfel de intervale sunt disjuncte $\implies f((a,b))=f((c,d)) \iff a+b=c+d$\\
        Daca $a+b=c+d\ si\ f((a,b))=f((c,d)) \implies \frac{(a+b)\cdot(a+b-1)}{2}+a = \frac{(c+d)\cdot(c+d-1)}{2}+c \iff a=c$, dar $a+b=c+d \implies b=d
        \implies f$ este injectiva. \hfill (1)\\\\
        Fie $x \in \mathbb{N} \implies \exists k\in \mathbb{N}\ a.i.\ x\in [\frac{k(k-1)}{2},\frac{k(k+1)}{2})$. Notam $y=x-\frac{k(k-1)}{2} \implies f((y,k-y))=
        x \implies \forall x \in \mathbb{N}\ \exists (y,k-y) \in \mathbb{N} \times \mathbb{N}\ a.i.\ f((y,k-y))=x \implies f$ este surjectiva \hfill (2)\\
        Din (1) si (2) $\implies f$ este bijectiva $\implies \mathbb{N} \times \mathbb{N}$ este numarabila \hfill \QEDA\\ \\
        Fie $\mathbb{Q}_+=\set{\frac{p}{q};p,q\in\mathbb{N}}$ si $g:\mathbb{N} \times \mathbb{N} \rightarrow \mathbb{Q}_+,\ g((a,b))=\frac{a}{b}$ o functie
        evident bijectiva $\implies \mathbb{Q}_+$ este echivalenta cu $\mathbb{N} \times \mathbb{N} \implies \mathbb{Q}_+$ este numarabila. Analog se demonsteaza ca
        $\mathbb{Q}_- =\set{-\frac{p}{q};p,q\in\mathbb{N}}$ este numarabila $\implies \mathbb{Q}$ este numarabila.
\end{document}

