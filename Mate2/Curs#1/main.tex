\documentclass{article}
\usepackage[utf8]{inputenc}
\usepackage{mathtools, nccmath}
\usepackage{mathabx}
\usepackage{mathrsfs}
\usepackage{natbib}
\usepackage{graphicx}
\usepackage{cancel}
\usepackage{upquote}
\usepackage{pgfplots}
\usepackage{amssymb}
\pgfplotsset{width=10cm,compat=1.9}
\usepgfplotslibrary{external}
\tikzexternalize
\usepackage{hyperref}
\usepackage{xparse}
\usetikzlibrary{through}
\usepackage{epigraph}

\usepackage{tikz}
\usetikzlibrary{angles,quotes}

\newcommand{\handC}{\mathscr{C}}

\title{Mate 1: Curs \#1}
\author{Profesor: Iulian Duca}
\date{30 Septembrie 2019}

\DeclarePairedDelimiterX{\set}[1]{\{}{\}}{\setargs{#1}}
\NewDocumentCommand{\setargs}{>{\SplitArgument{1}{;}}m}
{\setargsaux#1}
\NewDocumentCommand{\setargsaux}{mm}
{\IfNoValueTF{#2}{#1} {#1\,\delimsize|\,\mathopen{}#2}}%{#1\:;\:#2}

\makeatletter
\renewcommand*\env@matrix[1][*\c@MaxMatrixCols c]{%
    \hskip -\arraycolsep
    \let\@ifnextchar\new@ifnextchar
    \array{#1}}
\makeatother

\newcommand{\mytilde}{\raise.17ex\hbox{$\scriptstyle\mathtt{\sim}$}}

\begin{document}
	\epigraph{``Inteligenta artificiala nu va egala niciodata prostia naturala."}{--- \textup{Duca Iulian}}
    
    \maketitle
    \section{Matrici}
        Fie $A = \set{1,2,3,...,n}$ si $B = \set{1,2,3,...,m}, m,n \in \mathbb{N}^*$.\\
        Se numeste matrice cu $n$ linii si $m$ coloane, orice aplicatie $f : A\times B \rightarrow I$, unde $(I,+,\cdot)$ inel.\\
        Vom nota $f(i,j) = C_{ij},\ C = (C_{ij}), 1\leq i\leq n,\ 1\leq j\leq m$.\\
        Consideram matricea extinsa: 
        \begin{pmatrix}[cccc|c]
            x_{11} & x_{12} & \dots & x_{1n} & b_1 \\
            x_{21} & x_{22} & \dots & x_{2n} & b_2 \\
            \vdots & \vdots & \ddots & \vdots & \vdots \\
            x_{m1} & x_{m2} & \dots & x_{mn} & b_m \\
        \end{pmatrix}\\
        a sistemului $A \cdot X = B,\ X = 
        \begin{pmatrix}
            x_1\\
            x_2\\
            \vdots\\
            x_n\\
        \end{pmatrix},\ 
        B = \begin{pmatrix}
            b_1\\
            b_2\\
            \vdots\\
            b_n\\
        \end{pmatrix},\ A \in M_n(\mathbb{R})$\\
        Se observa ca daca se inverseaza doua linii pe matricea extinsa, se obtine o matrice extinsa a unui sistem echivalent cu cel initial.\\
        Daca se inmulteste o linie cu un numar nenul si se adauga la alta linie, rezulta un sistem echivalent cu cel original.\\
        Daca folosim cele 2 operatii enuntate anterior $\implies (A|B) \mytilde \dots \mytilde (I_n|B^*)$\\ \\
        Observatie:\\
        Daca avem un sistem $(A|C1)$ si un alt sistem $(A|C2)$, se pot rezolva simultan cele 2 sisteme cu metoda Gauss-Jordan:\\
        $(A|C1|C2)\mytilde \dots \mytilde (I_n|C_1^*|C_2^*)$\\
        $(A|C1) \mytilde \dots \mytilde (I_n|C_1^*) \iff C_1^* = A^{-1} \cdot C_1$
        Daca rezolva simultan $n$ sisteme de ecuatii de forma $\begin{pmatrix} 1\\0\\ \vdots\\ 0 \end{pmatrix}, \begin{pmatrix} 0\\1\\ \vdots\\ 0 \end{pmatrix}
            \dots \begin{pmatrix} 0\\0\\ \vdots\\ 1 \end{pmatrix}$\\
                $\begin{pmatrix}[c|c|c|c|c] A&\begin{pmatrix} 1\\0\\ \vdots\\ 0 \end{pmatrix}& \begin{pmatrix} 0\\1\\ \vdots\\ 0 \end{pmatrix}& \dots &
                    \begin{pmatrix}0\\0\\ \vdots\\ 1\end{pmatrix}\end{pmatrix} \iff (A|I_n) \mytilde \dots \mytilde (I_n|A^{-1})$ \\ \\
        Pentru determinarea matricelor echivalente cu matricea initiala se poate proceda astfel:
        \begin{enumerate}
            \item Daca $A_{11}\neq 0$ se imparte linia 1 la $A_{11}$ obtinand pe aceasta pozitie 1. Daca se inmulteste linia 1 cu $-A_{21}$ si se aduna la linia 2 se obtine pe linia $A_{21} = 0$. Se procedeaza analog pana cand sub elementul $A_{11}$ sunt numai valori $0$. Ceea ce s-a facut cu $A_{11}$ se face si cu elementele $A_{22}, A_{33}, \dots ,A_{nn}$, obtinandu-se astefel pe diagonala principala numai valori $1$ si sub aceasta numai valori $0$.
            \item Absolut analog cu pasul $1$, incepand cu coloana $n$ se obtin valori egale cu $0$ deasupra diagonalei principale.
        \end{enumerate}
        De remarcat este faptul ca in toate operatiunile descrise mai sus participa si elementele din coloana termenilor liberi. \\
		Exemplu: \\
		$\begin{pmatrix}[ccc|ccc]
            0 & 1 & 1 & 1 & 0 & 0 \\
            1 & 0 & 1 & 0 & 1 & 0 \\
            1 & 1 & 0 & 0 & 1 & 0 \\
        \end{pmatrix} \mytilde \begin{pmatrix}[ccc|ccc]
            1 & 0 & 1 & 0 & 1 & 0 \\
            0 & 1 & 1 & 1 & 0 & 0 \\
            1 & 1 & 0 & 0 & 0 & 0 \\
        \end{pmatrix} \mytilde \begin{pmatrix}[ccc|ccc]
            0 & 1 & 1 & 1 & 0 & 0 \\
            1 & 0 & 1 & 0 & 1 & 0 \\
            1 & 1 & 0 & 0 & 1 & 0 \\
        \end{pmatrix} \mytilde \begin{pmatrix}[ccc|ccc]
            1 & 0 & 1 & 0 & 1 & 0 \\
            0 & 1 & 1 & 1 & 0 & 0 \\
            0 & 1 & -1 & 0 & 1 & 1 \\
        \end{pmatrix} \mytilde \\ \\ \\ \mytilde \begin{pmatrix}[ccc|ccc]
            1 & 0 & 1 & 0 & 1 & 0 \\
            0 & 1 & 1 & 1 & 0 & 0 \\
            0 & 0 & -2 & -1 & -1 & 1 \\
        \end{pmatrix} \mytilde \begin{pmatrix}[ccc|ccc]
            1 & 0 & 1 & 0 & 1 & 0 \\
            0 & 1 & 1 & 1 & 0 & 0 \\
            0 & 0 & 1 & \frac{1}{2} & \frac{1}{2} & -\frac{1}{2} \\
        \end{pmatrix} \mytilde \begin{pmatrix}[ccc|ccc]
            1 & 0 & 1 & 0 & 1 & 0 \\
            0 & 1 & 0 & \frac{1}{2} & -\frac{1}{2} & \frac{1}{2} \\
            0 & 0 & 1 & \frac{1}{2} & \frac{1}{2} & -\frac{1}{2} \\
        \end{pmatrix} \mytilde \\ \\ \\ \mytilde \begin{pmatrix}[ccc|ccc]
            1 & 0 & 0 & -\frac{1}{2} & \frac{1}{2} & \frac{1}{2} \\
            0 & 1 & 0 & \frac{1}{2} & -\frac{1}{2} & \frac{1}{2} \\
            0 & 0 & 1 & \frac{1}{2} & \frac{1}{2} & -\frac{1}{2} \\
        \end{pmatrix} = (I_{n} | A^{-1})$ \\
        Se verifica usor ca matricea obtinuta este intradevar $A^{-1}$, intrucat respecta relatia $A \cdot A^{-1}=I_{3}$.
        \section{Spatii vectoriale}
        Fie $(V, +)$ un grup abelian si $(K, +, \cdot)$ un corp. Spunem ca V are structura de spatiu vectorial peste corpul K $\iff$
        	\begin{enumerate}
        		\item $\lambda(v + u)= \lambda v + \lambda u\ \forall \lambda, u \in K, v \in V$
        		\item $(\lambda + u)v = \lambda  v + uv\ \forall \lambda, u \in K, v \in V$
        		\item $ \lambda u \left(v\right) = \lambda \left(uv\right)$
        		\item $1 \cdot v = v, \forall \ v \in V$
        	\end{enumerate}
        Elementele din $V$ se numesc vectori iar elementele din $K$ se numesc scalari.\\
        Exemple de spatii vectoriale:
        \begin{itemize}
        	\item $V = \mathbb{R}^{n}, K = \mathbb{R}, n \in \mathbb{N}$
        	\item $V = \mathbb{C}, K = \mathbb{R}$
        	\item $V = M_{n,m}(\mathbb{R}), K = \mathbb{R}$
        	\item $V = V_{3}, K = \mathbb{R}$
        	\item $V = \mathbb{R}[X], K = \mathbb{R}$
        \end{itemize}
        Fie scalarii $\alpha_{1}, \alpha_{2}, \dots, \alpha_{n}$ si vectorii $v_{1}, v_{2}, \dots v_{n}$. Atunci expresia $\alpha_{1} \cdot v_{1} + \dots + \alpha_{n} \cdot v_{n}$ se numeste combinatie liniara a vectorilor $v_{1}, v_{2}, \dots v_{n}$, cu scalarii $\alpha_{1}, \alpha_{2}, \dots, \alpha_{n}$. \\ \\
        Definitie: Fie $V$ un spatiu vectorial peste $K$. Vectorii $\set{ v_{1}, v_2, \dots, v_{n}}$ formeaza sistem de generatori pentru $V$ $\iff \forall\ v \in V\ \exists\ \alpha_{1}, \alpha_{2}, \dots , \alpha_{n}$ a.i. $v = \alpha_{1} \cdot v_{1} + \dots + \alpha_{n} \cdot v_{n}$. \\ \\
        Definitie: Vectorii $v_{1}, v_{2}, \dots v_{n}$ din spatiul vectorial $V$ peste $K$ formeaza un sistem liniar independent peste $K$ daca din orice relatie $\alpha_{1} \cdot v_{1} + \dots + \alpha_{n} \cdot v_{n} = 0 \implies \alpha_{1} = \alpha_{2} = \dots = 0$. Altfel spus, un sistem de generatori nu poate fi scris ca o combinatie liniara decat decat daca toti scalarii acesteia sunt nuli. \\ \\
        Definitie: Un spatiu vectorial care admite un sistem de generatori cu numar finit de vectori se numeste spatiu vectorial finit generat. \\ \\
        Definitie: O multime de vectori care formeaza un sitem de generatori liniar independenti se numeste baza. \\ \\
        Proprietate: Din orice sistem de generatori se poate extrage o baza.\\
        Demonstratie: Vectorii din sistemul de generatori care se pot scrie ca fiind o combinatie liniara a celorlalti vectori se pot exclude din sistemul de generatori. Procedeul poate sa fie continuat pana in momentul in care nici unul dintre vectori nu mai poate fi scris ca o combinatie liniara a celorlalti.
\end{document}

